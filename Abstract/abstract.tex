% ************************** Thesis Abstract *****************************
% Use `abstract' as an option in the document class to print only the titlepage and the abstract.
\begin{abstract}
	 
New technological developments have strongly influenced the way we live our
everyday lives, the way we perform daily activities such as shopping or 
communication. Every day, more and more people have access to the Internet
and to newest technological tools which help them make their lives easier
and of better quality. This is something that since the past ten years
Governments of different countries have tried to take advantage of, by 
applying all these new tools to their different functions in order to 
simplify their services, make them globally accessible and reduce the
red tape and generally become more effective both internally as well 
as externally. 

Countries such as Estonia, the UK, Singapore or Denmark have managed
to migrate to a fully digital government where every operation such as
tax paying, transportation and drive license issuing, employing people
and so on are done electronically.On the other hand many other countries
like Greece are now trying to follow the example set by the aforementioned
countries by implementing step by step services.

As mentioned by some representatives of the GOV UK team while at workshop
at National Infrastructure for Research and Technology the biggest
challenge that every government faces and must solve in the process
of becoming an e-government is the change of culture as well as to
integrate all this services so they share the information and update
each other dynamically depending on the user's actions.

However, this is easier said than done. E-Government infrastructures
are used every day by millions of civilians and thus need to be easy
to navigate, pretty. Moreover, the need to be secure, efficient and
easy to use. The backend design as well as the User Interface / User
Experience must be orchestrated in such as way that makes the user come
back.


\end{abstract}
