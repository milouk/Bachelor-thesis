% ************************** Thesis Abstract *****************************
% Use `abstract' as an option in the document class to print only the titlepage and the abstract.
\begin{abstract}
	 
New technological developments have strongly influenced the way we live our
everyday lives, the way we perform daily activities such as shopping or 
communication. Every day, more and more people have access to the Internet
and to newest technological tools which help them make their lives easier
and of better quality. This is something that since the past twenty years
Governments of different countries have tried to take advantage of, by 
applying all these new tools to their different functions in order to 
simplify their services, make them globally accessible and reduce the
red tape and generally become more effective both internally as well 
as externally. 

Countries such as Estonia, the UK, Singapore or Denmark have managed
to migrate to a fully digital government where every operation such as
tax paying, transportation and drive license issuing, employing people
and more are done digitally.

This study provides an analysis of Web APIs as enablers for the digital
transformation of government. While digital transformation of government
is much wider than the technologies which can potentially support it, an
analysis of the role of Web APIs in the public sector is highly relevant
to illustrate how technology can enable the transformation of government.
The aim of this work has been to identify the ability of Web APIs to assist
Member States with enabling their digital transformation. Areas of specific
focus include cross-border interoperability between Member States and the 
opportunity for the EU to become involved in developing or advocating API 
standards. 

This research set out to explore the API landscape in the EU public sector.
API is the acronym for Application Programming Interface and it refers 
to a set of clearly defined methods of communication between a service 
and any other software or components,essentially, a software intermediary
that allows two applications to interact with each other. The purpose of
the study has been to identify the ability of Web APIs (hereafter “APIs”)
to assist Member States with enabling their digital transformation. Areas
of specific focus include aspects such as cross-border interoperability
between Member States, and the opportunity for the EU to become involved
in developing or advocating API standards.




\end{abstract}
