% ************************** Thesis Abstract *****************************
% Use `abstract' as an option in the document class to print only the titlepage and the abstract.
\begin{abstract}
	 
Recent technological developments strongly influence the way we lead our every - day life and the way we perform daily activities as shopping and communication. More and more new users are getting access to the Internet and to the newest technological tools, in order to make their life easier and better.

Having perceived the advantages of the use of new technology tools, many Governments worldwide have consistently, for the last twenty years, introduced new tools into state operational structure, in order to simplify their public services, make them widely accessible to the public, reduce the red tape, offer more efficient services and therefore a more effective governance. Countries, like Estonia, the UK, Singapore and Denmark have successfully established fully digitalised public services, including tax paying, transportation, drive licence issuing, and employment procedures.

This research set out to explore the Application Programming Interface (API) landscape in the EU public sector and how APIs could play a role into the digital transformation of governance.  More specifically, the aim of this work has been to examine the ability of Web APIs (hereafter “APIs”) to assist Member States of the EU into enabling their digital transformation. Areas of specific focus include cross-border interoperability between Member States and the opportunity for the EU to become involved in developing or advocating API standards.

From a technological point of view Application Programming Interface (API) refers to a set of clearly defined methods of communication between a service and any other software or components. It is essentially a software intermediary that allows two applications to interact with each other. In the public sector, APIs enable important functionality and information held in one agency’s system or department to be readily available to another without significant and expensive development effort. As well as cross-departmental (agency) access to functionality and information (or even cross-border with a different country’s administration) APIs also provide the ability to share information and functionality more widely, i.e. to developers and ultimately to citizens for consumption through web or mobile based applications. 

As APIs enable the public sector to create ‘ecosystems’ inside and outside a government system, overcome the restrictions of traditional integration solutions, facilitate the opening of large data sources to citizens and other third parties and support innovation, they seem to play a key role in the digital transformation of efficient governance, generating benefits for the citizen, for business and for the economy.


\end{abstract}
