%!TEX root = ../thesis.tex
% ******************************* Thesis Appendix A ****************************
\chapter{API Types} 

% Note: It may be necessary to compile the document several times to get a multi-page table to line up properly
\begin{longtable}[c]{|l|l|l|}
	\hline
	\multicolumn{1}{|c|}{\textbf{API Type}} & \multicolumn{1}{c|}{\textbf{Data Formats}} & \multicolumn{1}{c|}{\textbf{Description}} \\ \hline
	\endfirsthead
	%
	\endhead
	%
	\multirow{5}{*}{Web APIs} & SOAP over HTTP/S & \begin{tabular}[c]{@{}l@{}}SOAP is a protocol that defines the\\ communication method, and the\\ structure of the messages. The\\ data transfer format is XML.\\ A SOAP service publishes a definition\\ of its interface in a machine-readable\\ document, using WSDL – Web Services\\ Definition Language.\end{tabular} \\ \cline{2-3} 
	& XML-RPC over HTTP/S & \begin{tabular}[c]{@{}l@{}}XML-RPC is an older protocol than\\ SOAP. It uses a specific XML format\\ for data transfer, whereas SOAP allows\\ a proprietary XML format. An XMLRPC\\ call tends to be much simpler, and to use\\ less bandwidth, than a SOAP call.\end{tabular} \\ \cline{2-3} 
	& \begin{tabular}[c]{@{}l@{}}JSON- RPC over\\ HTTP/S\end{tabular} & \begin{tabular}[c]{@{}l@{}}JSON-RPC is similar to XML-RPC,\\ but uses JSON instead of XML for\\ data transfer.\end{tabular} \\ \cline{2-3} 
	& REST over HTTP/S & \begin{tabular}[c]{@{}l@{}}REST is not a protocol, but rather a\\ set of architectural principles. Some\\ of the characteristics required of a \\ REST service include simplicity of\\ interfaces, identification of resources\\ within the request, and the ability to\\ manipulate the resources via the interface.\\ The most commonly-used data format is\\ JSON or XML. Often the service will offer\\ a choice, and the client can request one or\\ the other by including “json” or “xml” in\\ the URL path or in a URL parameter.\\ In a well-defined REST service, there is no\\ tight coupling between the REST interface\\ and the underlying architecture of the\\ service. This is often cited as the main\\ advantage of REST over RPC\\ (Remote Procedure Call) architectures.\end{tabular} \\ \cline{2-3} 
	& GraphQL & \begin{tabular}[c]{@{}l@{}}GraphQL is a data query language\\ developed internally by Facebook in\\ 2012 before being publicly released\\ in 2015. It provides an alternative to\\ REST and ad-hoc webservice architectures.\\ While typical REST APIs require loading\\ from multiple URLs, GraphQL APIs get\\ all the data an app developer needs in a single\\ request enhancing speed of response even on\\ slow mobile network connections.\end{tabular} \\ \hline
	\begin{tabular}[c]{@{}l@{}}Library\\ based APIs\end{tabular} & \begin{tabular}[c]{@{}l@{}}JavaScript APIs,\\ TWAIN, Twilio\end{tabular} & \begin{tabular}[c]{@{}l@{}}To use this type of API, an application\\ will reference  or import a library of\\ code or of binary functions, and use the\\ functions/routines from that library to\\ perform actions and exchange information.\end{tabular} \\ \hline
	\begin{tabular}[c]{@{}l@{}}Class-based\\ APIs (object\\ oriented) – a\\ special type of\\ library based API\end{tabular} & Java API & \begin{tabular}[c]{@{}l@{}}These APIs provide data and functionality\\ organised around classes, as defined in \\ objectoriented languages. Each class offers \\ a discrete set of information and associated\\ behaviours, often corresponding to a human\\ understanding of a concept.\end{tabular} \\ \hline
	\begin{tabular}[c]{@{}l@{}}Object\\ remoting\\ APIs\end{tabular} & CORBA & \begin{tabular}[c]{@{}l@{}}These APIs use a remoting protocol, such as\\ CORBA – Common Object Request Broker\\ Architecture. Such an API works by\\ implementing local proxy objects to represent\\ the remote objects, and interacting with the\\ local object. The same interaction is then\\ duplicated on the remote object,via the\\ protocol.\end{tabular} \\ \hline
\end{longtable}