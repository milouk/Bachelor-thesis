%!TEX root = ../thesis.tex
%*******************************************************************************
%*********************************** First Chapter *****************************
%*******************************************************************************

\chapter{Introduction}  %Title of the First Chapter

\ifpdf
    \graphicspath{{Chapter1/Figs/Raster/}{Chapter1/Figs/PDF/}{Chapter1/Figs/}}
\else
    \graphicspath{{Chapter1/Figs/Vector/}{Chapter1/Figs/}}
\fi

At a strategy level, the Tallinn Declaration, signed on the 6th October 2017
~\citep{tallin_declaration}, confirms the commitment to the vision set out in the EU eGovernment
Action Plan 2016-2020~\citep{action_plan} and in the European Interoperability Framework (EIF)~\citep{eif}.
In a time span of five years (2018-2022), steps will be taken towards
the implementation of the following principles in EU open organizations: "computerized
naturally, comprehensiveness furthermore, availability", "once-just", "dependability
and security", "transparency and straightforwardness", and "interoperability as a
matter of course", just as national interoperability structures dependent on the
European Interoperability Framework (EIF).

In the Declaration the “user-centricity principles for design and delivery of
digital public services” is vital. Public administrations and digital public services,
upon interaction must fulfill the following requirements: digital interaction, accessibility, security, availability
and usability, reduction of the administrative burden, digital delivery of public services,
citizen engagement, incentives for digital service use, protection of personal data and
privacy, redress and complaint mechanisms. At the same time, the Communication on “Building the data economy” (COM (2017) 9)
looks at proven or potential blockages to the free movement of data and presents
options to remove unjustified and or disproportionate data location restrictions
in the EU. It also considers the barriers around access to, and transfer of,
non-personal machine-generated data and data liability, as well as issues related
to the portability of non-personal data, interoperability and standards.
In particular, it aims at the development of technical solution for
the reliable exchange and identification of data. 

Moreover, the digital transformation of society, business and government is raising
issues for a range of policy matters in the EU. As e-government has
been in place for the last 20 years, it is timely to explore the interaction between
technology and government activities from the perspective of digital government.
To understand the intertwined forces that play a role in this transformation process,
and their dynamics, 

\textbf{contributions from disparate fields and discourses on this topic
are being contrasted and compared.}

Aspects of the digital
transformation of government, concern the use of Web Application Programming Interfaces
(hereafter called “APIs”)). APIs can be seen as “safe entry ports for new and
innovative uses of data” held by companies and potentially, public administrations.

An opportunity exists to comprehend the current context and use of this technology
early in the innovation cycle of e-government in EU countries.

This study is set to explore
Web APIs as enablers for the digital transformation of governments. While
digital transformation of government is much wider than the technologies which
can potentially support it, an analysis of the role of APIs in the public sector
is highly relevant to illustrate how technology can enable transformation of
government. This study examines APIs, and their role in the EU public sector as
well as it points out the differences with the private sector and explores future
trends with a particular focus on current use of APIs in projects developed by various
EU Countries.

\clearpage

%********************************** %First Section  **************************************
\section{Glossary} %Section - 1.1 

% Note: It may be necessary to compile the document several times to get a multi-page table to line up properly
\begin{longtable}[c]{|l|l|}
	\hline
	\multicolumn{1}{|c|}{\textbf{Term}} & \multicolumn{1}{c|}{\textbf{Definition}} \\ \hline
	\endfirsthead
	%
	\endhead
	%
	API & \begin{tabular}[c]{@{}l@{}}Application Programming Interface - It is a set of clearly\\ defined methods of communication between the service\\ and any other software or components.\end{tabular} \\ \hline
	API Ecosystem & \begin{tabular}[c]{@{}l@{}}The developers, and the users of the application constructs\\ they build through an API, either within a company or on\\ the Internet with business partners, customers, citizens etc.\end{tabular} \\ \hline
	API Economy & \begin{tabular}[c]{@{}l@{}}A set of business models and channels — based on secure\\ access of functionality and exchange of data to an ecosystem\\ of developers and the users of the app constructs they\\ build — through an API, either within a company or on the\\ Internet with business partners, customers, citizens etc.\end{tabular} \\ \hline
	API Versioning & \begin{tabular}[c]{@{}l@{}}The ability to change without rendering older versions of\\ the same API inoperable.\end{tabular} \\ \hline
	\begin{tabular}[c]{@{}l@{}}API\\ Standardisation\end{tabular} & \begin{tabular}[c]{@{}l@{}}A uniform way for APIs to be expressed and consumed,\\ from COM and CORBA object brokers to web services\\ to today’s RESTful patterns.\end{tabular} \\ \hline
	\begin{tabular}[c]{@{}l@{}}API information\\ control\end{tabular} & \begin{tabular}[c]{@{}l@{}}A built-in means for enriching and handling the\\ information embodied by the API. This information\\ includes metadata, approaches to handling batches of\\ records, and hooks for middleware platforms, message\\ brokers, and service buses. It also defines how APIs\\ communicate, route, and manipulate the information\\ being exchanged.\end{tabular} \\ \hline
	API portal & \begin{tabular}[c]{@{}l@{}}A means for developers to discover, collaborate,\\ consume, and publish APIs. To support the overall\\ goal of self-service, these portals describe APIs in\\ a way that represents their functionality, context\\ (the business semantics of what they do, and how \\ they do it), non-functional requirements (scalability,\\ security, response times, volume limits, and resiliency\\ dimensions of the service), versioning, and metrics \\ tracking usage, feedback, and performance.\\ For organizations without mature master data or\\ architectural standards, the API portal can still offer\\ visibility into existing APIs and provide contact information\\ for individuals who can describe features, functions, and\\ technical details of services.\end{tabular} \\ \hline
	API gateway & \begin{tabular}[c]{@{}l@{}}A mechanism that allows consumers to become\\ authenticated and to “contract” with API specifications\\ and policies that are built into the API itself. Gateways\\ make it possible to decouple the “API proxy”—the node\\ by which consumers logically interact with the service—from\\ the underlying application for which the actual service is \\ being implemented. The gateway layer may offer the means\\ to load balance and throttle API usage.\end{tabular} \\ \hline
	API brokers & \begin{tabular}[c]{@{}l@{}}Enrichment, transformation, and validation services\\ to manipulate information coming to/from APIs, as\\ well as tools to embody business rule engines, workflow,\\ and business process orchestration on top of underlying APIs.\end{tabular} \\ \hline
	\begin{tabular}[c]{@{}l@{}}API\\ management\\ and monitoring\end{tabular} & \begin{tabular}[c]{@{}l@{}}A centralized and managed control level that provides\\ monitoring, service level management, SDLC process\\ integration, and role-based access management across\\ all three layers above. It includes the ability to instrument\\ and measure API usage, and even capabilities to price and\\ bill charge-back based on API consumption—to internal,\\ or potentially external, parties.\end{tabular} \\ \hline
	RESTful API & \begin{tabular}[c]{@{}l@{}}REST stands for “representational state transfer.” APIs\\ built according to REST architectural standards are\\ stateless and offer a simpler alternative to some SOAP\\ standards. For example, REST enables plain-text exchanges\\ of data assets instead of using complex WSDL protocols.\\ It also makes it possible to inherit security policies\\ from an underlying transport mechanism. At a high level,\\ these and other simplified approaches can deliver better\\ performance and faster paths to develop, deploy, and\\ triage.\end{tabular} \\ \hline
	\caption{Glossary}
	\label{glossary}\\
\end{longtable}

%********************************** %Second Section  *************************************
\section{API Overview} %Section - 1.2


APIs have become a key technological component of modern digital
architectures, impacting every sector of the global economy. In the public
sector specifically, APIs are a fundamental enabler of the transformation of its
operations from analogue(manual, paper) to digital.

The purpose of this study is to showcase the 
major contribution of APIs, when member states are in
pursuit of their digital transformation.
In order to explore this purpose, our investigation has
covered the following topics:

\begin{itemize}
	\item The current use of APIs in the EU public sector.
	\item Differences between API use in the public and the private sector.
	\item The future trends for APIs.
	\item Aspects of the API Landscape including API Ecosystems, API as low complexity infrastructures,
	API as components of a business plan.
\end{itemize}
                                                                       
In summary, in this study web based research has been used as well as my
experience from my internship in GRNET to gather information for analysis
of successful but diverse API based case studies from a range of EU countries
and sectors.

\textbf{API Overview}

API interaction occurs when one application would like to:
\begin{itemize}
	\item Request a service from another application.
	\item Send data to that application.
	\item Access or query the data held by another application.
	\item Update data held in that application.

\end{itemize}

\textbf{Types of APIs}

APIs represent an architectural approach that revolves around providing
programmable interfaces to different applications. It is technology agnostic,
and creates a flexible, loosely coupled architecture that allows a solution to be
made up of components that can more easily be switched in and out. The API approach
is also a essential enabler for application developers to create apps that rapidly adapt
to end user needs~\citep{hcl_tech}. 

In the public sector, APIs enable important functionality and information held in
one agency’s system or department to be readily available to another without
significant and expensive development effort. As well as cross-departmental
access to functionality and information (or even cross-border with a
different country’s administration) APIs also provide the ability to share
information and functionality more widely, i.e. to developers and ultimately
to citizens for consumption through web or mobile based applications.

Although there are many different types of API (see Appendix I), this study is
most concerned with Web APIs. Web services expose these APIs as endpoints that
any internet-enabled language or software can access, in exactly the same way
browsers access websites and services~\citep{fed_tech}. Web APIs deliver requests to the
service provider, and then deliver the response back to the requestor,
i.e. they are an interface for web applications, or applications that need to
connect to each other via the Internet to communicate~\citep{define_api}.

Web APIs themselves can be broken down further based on the type of data format
that they harness, for example, well known types are Simple Object Access Protocol
(SOAP), Remote Procedure Call (RPC) based APIs, and the Representational State
Transfer (REST) architectural style. GraphQL – is a data query language growing
in popularity and has been adopted by leading social media outlets such as Facebook
and Pinterest~\citep{graphql} as a type of API. While typical REST APIs require loading
from multiple URLs, GraphQL APIs get all the data an app developer needs in a single
request enhancing speed of response even on slow mobile network connections~\citep{graphql}.

Whilst the more traditional APIs are used as integration points within systems
hidden from view, Web APIs are often publicly available and can be ‘advertised’
via API Directory sites online. Tens of thousands~\citep{programmableweb_search} are available for developers
to deliver consumable information to end users to do everything, from checking
traffic and weather, to updating a social media status, or even to make payments.

In the geospatial domain, besides existing private companies famous API proposals
(e.g. Google Map), the Open Geospatial Consortium (OGC) has created standards to
support the exchange of geospatial information~\citep{opengeospatial}. They describe their Web services
API standards as an agreed specification of rules and guidelines about how to
implement software interfaces and data encodings~\citep{opengeospatial_elearning}. Geospatial software vendors,
developers and users collaborate in the OGC’s consensus process to develop and
agree on standards that enable information systems to exchange geospatial
information and instructions for geoprocessing. OGC standards are open standards.
The OGC interface standards are also available in the REST style, and cover a
number of aspects:
\begin{itemize}
	\item Visualisation standards e.g. Web Map Service (WMS).
	\item Data Access Standards e.g. Web Feature Service (WFS), SensorThings API.
	\item Processing Standards e.g. Web Processing Service (WPS).
	\item Metadata and Catalogue Service Standards e.g. Catalog Service for the Web (CWS).
	\item The informatics contract between the client code which manipulates normalized
	data structures of geographic information based on the published API and the library
	code, e.g. the GeoAPI Implementation Standard.
\end{itemize}

The standards above are part of the few globally agreed specifications adopted by
the Technical Committee 211 of the International Organization for Standardization
(ISO). ISO is also known to be working on standards in other sectors, notably in
Financial Services with ISO 20022~\citep{iso}: however, because of they are work in
progress,
details about them are still limited. Whilst standards of this formal and specific
nature are used in the EU, there is clear evidence that the need for harmonizing
APIs lifecycle has been recognized. For example, the UK Government Digital Service
recognized that departments were developing APIs using different tools, platforms
and approaches~\citep{tech_gov}, and have set about working with industry to create a set of
common principles for API design. The output has been a set of guidelines on how
developers working in any UK public sector organisation should build APIs~\citep{gov_uk_api} to
ensure consistency, and success. These guidelines apply to other countries too. Specifically,
Greece has implemented these guidelines in some of its public sector applications such as the digital
Solemn Declaration / Authorization issuing service.
Although they are titled as a ‘standard’, they
are generic, and not exact or specific in the way that an ISO or OGC standard is.
Nevertheless, given the fact that government is increasingly using APIs to
automate processes and provide citizens with access to new services it is hoped
this approach will make integration simpler and faster.