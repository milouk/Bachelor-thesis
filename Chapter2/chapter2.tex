%!TEX root = ../thesis.tex
%*******************************************************************************
%****************************** Second Chapter *********************************
%*******************************************************************************

\chapter{APIs in the Public Sector}

\ifpdf
    \graphicspath{{Chapter2/Figs/Raster/}{Chapter2/Figs/PDF/}{Chapter2/Figs/}}
\else
    \graphicspath{{Chapter2/Figs/Vector/}{Chapter2/Figs/}}
\fi

The Internet, social media, smartphones, and access to real-time information
have not only made people’s daily lives easier, but have changed citizens’
expectations of how products and services are delivered. In the public sector,
this shift has raised the expectations of citizens and business in their
interactions with government.

People are demanding transparency, accountability, access to information and
competent service delivery from their governments. They also expect policies
and services to be tailored to their needs and address their concerns.

In this section, we will explore how APIs are used in the public sector.
We will firstly look at typical uses, such as the enablement of ecosystems,
before looking at some specific examples of API use. In addition, we will
cover some challenges and considerations, and examine data on the APIs
advertised in one of the most respected API Directories, (ProgrammableWeb)
as a further indicator of the way APIs are used in the public sector.


\section{APIs enable the public sector to create ‘ecosystems’}

API based ecosystems can be defined as the extended interrelationships enabled
by developers who create applications that link various groups of stakeholders
to each other via API based solutions that use the internet to communicate [cite 18].

An ecosystem may be created within a government agency, between agencies, or
it may be wider reaching, for example between a government and another government
or between a government, their citizens, and potentially third party providers.


% arctext from Andrew code with modifications:
%Variables: 1: ID, 2:Style 3:box height 4: Radious 5:start-angl 6:end-angl 7:text {format along path} 
\def\arctext[#1][#2][#3](#4)(#5)(#6)#7{
	
	\draw[
	color=white,
	thick,
	line width=1.3pt,
	fill=#2
	]
	(#5:#4cm+#3) coordinate (above #1) arc (#5:#6:#4cm+#3)
	-- (#6:#4) coordinate (right #1) -- (#6:#4cm-#3) coordinate (below right #1) 
	arc (#6:#5:#4cm-#3) coordinate (below #1)
	-- (#5:#4) coordinate (left #1) -- cycle;
	\def\a#1{#4cm+#3}
	\def\b#1{#4cm-#3}
	\path[
	decoration={
		raise = -0.5ex, % Controls relavite text height position.
		text  along path,
		text ={#7},
		text align = center,        
	},
	decorate
	]
	(#5:#4) arc (#5:#6:#4);
}

%arcarrow, this is mine, for beerware purpose...
%Function: Draw an arrow from arctex coordinate specific nodes to another 
%Arrow start at the start of arctext box and could be shifted to change the position
%to avoid go over another box.
%Var: 1:Start coordinate 2:End coordinate 3:angle to shift from acrtext box  
\def\arcarrow[#1](#2)(#3)[#4]{
	\draw[thick,-,>=latex,color=#1,line width=1pt,shorten >=-2pt, shorten <=-2pt] 
	let \p1 = (#2), \p2 = (#3), % To access cartesian coordinates x, and y.
	\n1 = {veclen(\x1,\y1)}, % Distance from the origin
	\n2 = {veclen(\x2,\y2)}, % Distance from the origin
	\n3 = {atan2(\y1,\x1)} % Angle where acrtext starts.
	in (\n3-#4: \n1) -- (\n3-#4: \n2); % Draw the arrow.
}


\begin{figure}
	\begin{center}
	\begin{tikzpicture}[
	% Environment Cfg
	font=\sf    \scriptsize,
	% Styles
	myarrow/.style={
		thick,
		-latex,
	},
	Center/.style ={
		circle,
		fill=white,
		text=root,
		align=center,
		font =\footnotesize,
		inner sep=1pt,          
	},
	]
	
	% Drawing the center
	\node[Center](ROOT) at (0,0) {Governemnt \\ APIs};
	
	% Drawing the Tex Arcs
	
	% \Arctext[ID][box-style][box-height](radious)(start-angl)(end-angl){|text-styles| Text}
	% Node 1:   
	\arctext[N1][black][30pt](3)(270)(90){|\footnotesize\bf\color{white}| Private};
	%Sub 1:
	\arctext[N1S1][black][26pt](5)(270)(90){|\footnotesize\bf\color{white}| Agency Systems };

	%Node 2:
	\arctext[N2][black][30pt](3)(-90)(0){|\footnotesize\bf\color{white}| Open Secured};
	%Sub 1:
	\arctext[N2S1][black][26pt](5)(90)(65){|\footnotesize\bf\color{white}| Developer Networks};

	%Sub 2:
	\arctext[N2S2][black][26pt](5)(65)(30){|\footnotesize\bf\color{white}| Commercial Developers};

	%Sub 3:
	\arctext[N2S3][black][26pt](5)(30)(0){|\footnotesize\bf\color{white}| Partner Service Providers};

	%Sub 4: 
	\arctext[N2S4][black][26pt](5)(0)(-60){|\footnotesize\bf\color{white}| Governemnt Agencies};
	
	%Sub 5: 
	\arctext[N2S4][black][26pt](5)(-90)(-60){|\footnotesize\bf\color{white}| Business Unit Developers};
	
	%Node 3:
    \arctext[N3][black][30pt](3)(0)(90){|\footnotesize\bf\color{white}| Open Public};

	\end{tikzpicture}
\end{center}
\label{Figure 1}
\caption{Ecosystems enabled by government APIs} [cite 19]
\end{figure}

\clearpage

The figure above illustrates the way in which APIs are used, and the typical
ecosystem that they facilitate in the public sector.
\begin{itemize}
	\item Private – Agency Systems: These APIs are generally used to facilitate
	 the sharing of data between systems within an agency, avoiding the need
	 for complex point to point integration. They are not visible to any person
	 or body outside of the agency and are generally in the domain of the IT
	 department. An example maybe a link between an internal HR system and a
	 Payroll solution.
	 \item Open Public – At Large Developer Networks: Open APIs (i.e. you do
	 not require permission to access them) are the access point for developers
	 to access large public data sources such as a census information or other
	 similar statistical data, perhaps live sensor data from which to create
	 citizen-facing applications.
	 \item Open Public – Commercial Developers: As above, but developers who are
	 looking to gather freely available data for use, generally, in applications
	 that can be sold. They may add value by ‘mashing’ the data, i.e. combining
	 data on public transportation networks with location data available on an
	 individual’s smart phone to help the citizen make travel choices in real-time...
	 Because of this openness, third-party integration of software is not only easier
	 but less problematic. Developers have access to the API at all times, so they
	 can ensure that the two-way communication between assorted pieces of software
	 is correct, rather than having to guess at the appropriate methods to use.
	 
	 It is also worth noting the economic stimulation that this can bring. Transport
	 for London’s policy of working with major IT players (Google, Apple, Waze etc.)
	 but allowing their data to be available via the Open Government License has led
	 to the creation of additional economic activity in the order of £100m of direct
	 value and has enabled some 1,000 jobs [cite 20].
	 \item Open Public/Secured – Partner Service Providers: The APIs are open to
	 partners perhaps in the private sector which may include healthcare providers
	 for example, who in some member states are interested in sharing healthcare
	 records, or confirming eligibility for free or subsidized treatment based on
	 data held by a government agency.
	 \item Open Secured – Government Agencies: These APIs are available to other
	 government agencies and allow them to share data only once they have
	 authenticated. This supports many of the core tenets of digital government,
	 allowing agencies to collect data on a citizen only once, and then share it
	 securely. An example may involve the sharing of citizen data between say the
	 agency responsible for income and taxation, and those providing benefits in
	 order that eligibility could be confirmed. Please see later case studies
	 relating to Estonia X-Road, and Amsterdam City Data for more on this.
	 
	 Although not specifically mentioned in the diagram above, the ability to use
	 APIs is not constrained by sector or geographical boundaries. Open Secured
	 – Government Agencies could include an application to application link
	 between governments of different member states. A good example (explored
	 further later) would be the Estonian X-Road Platform which uses APIs to
	 share citizen’s healthcare information with Finland.
	 \item Open Secured – Business Unit Developers: Similar to the above, but
	 instead of basic inter-agency data sharing, in this case the data is being
	 consumed and then in some way supplemented in order to be useful by developers
	 within a government agency. They are used to create custom applications around
	 internal data assets for agency use.
\end{itemize}

In summary, the creation of an ‘ecosystem’ of providers and consumers fosters
openness and efficiency, and can also spawn the development of innovative service
models, some of which may lead to revenue generation for the agencies concerned
(for example mapping data [cite 21], or gazetteer data). Their ability to provide access
into the heart of government, in turn allows government to realise its objectives
of openness, and of delivering efficient, secure, transparent and interoperable
citizen centric services. The APIs are, therefore, a crucial technological
component, which will underpin empowering the evolution of public service delivery
models, enabling agencies to accelerate their transformation from eGovernment to
Digital Government.

\section{APIs enable public sector agencies to overcome complex integration}

Nearly all EU countries have developed their computing infrastructure over
many years, constructing a legacy of large, complex information systems featuring
interfaces to pass information from one system to another. The majority of these
interfaces were point to point and custom built to meet the needs of a particular
project or agency at a point in time. As the number of interfaces grew, so did the
maintenance burden; the inter-relationships and the data duplication leading to an
expensive, complex and inefficient architecture [cite 22]. In summary, these
“siloed”, legacy government systems and associated business processes increase
risk and exacerbate challenges in data sharing and service delivery across the
ecosystem.

APIs provide an opportunity, in effect a structural ‘workaround’, to enable the
information within these legacy systems to be exposed with comparably low complexity
and investment. They can be plugged into legacy systems of record such as ERP
systems [cite 23], or citizen facing records to make the data records directly
available, thus helping to bypass the complex interfaces of existing systems,
and allow data sharing to be accomplished more easily. This means that a
well-designed government ecosystem could help minimise the frequency that
citizens or businesses will have to provide the same information (Once Only
Principle, OOP).

A good EU example of where API infrastructure is currently being used to overcome
the restrictions of traditional integration solutions is Estonia’s X-Road Platform.
It allows citizens to provide common ‘private and sensitive’ information to public administrations only once, for example, marital status. The ecosystem also includes
private institutions such as banks who can have access in order to perform various
functions. X-Road is examined in more detail in the case studies section of this report.

\textbf{EU Example: ESTONIA X-ROADS PLATFORM}

“X-Road is the backbone of e-Estonia. Invisible yet crucial, it allows the
nation’s various public and private sector e-Service databases to link up and
function in harmony.” [cite 24]

X-Road is a government API framework developed by the Estonian government
and licensed under the MIT license. It is also used as a backbone of the Finnish
National Data Exchange Layer.Originally built for SOAP/XML web services, it now
extends to REST APIs. Rather than requiring governments to develop API management
directly, X-Road provides an API management layer, including an API gateway,
which is open-sourced and available to governments worldwide. [cite 25]

The X-Road solution includes a security server to provide identity and access
management for government API access. It also provides central monitoring of
API traffic. In addition to the management of APIs, it also provides an
aggregation layer in front of multiple databases. This facilitates the creation
and delivery of data access APIs.

Since each government service/agency has its own databases they all use X-Road
to securely communicate and share ‘private and sensitive’ data to protect the
‘once only’ principle of sharing data with government. The service also incorporates
many other sectors numbering over 900 organisations and enterprises including those
in the banking, health and utility sectors [cite 26]. Whilst they may use the platform to
perform functions such as identity verification, powerful use cases such as automated extraction of funds from bank accounts for those failing to keep up to date with
taxes are possible.

All that being said, the X-Road itself is a ‘very low level engineered application’
[cite 27]. Following certification, an organisation deploys an x-road gateway
so that it can hold secure private communications via APIs with other certified
organisations that are legally able to share data with it. As a collective toolset,
the e-Estonia services provide the government of Estonia and its partners,
including Finland, with a platform on which to innovate and use digital
transformation to deliver new services across the globe.

\section{APIs support the public sector open government initiatives}

Open Government can be defined as the opening up of government processes,
proceedings, documents and data for public scrutiny and involvement, and is
now considered as a fundamental element of a democratic society [cite 28].
The Open government initiative started in 2009 by Barak Obama [cite 29], after that,
numerous governments adopted open data initiatives. It is founded on the
belief that greater transparency and public participation can not only lead
to better policies and services, they can also promote public sector
integrity, which is essential to regaining the trust of citizens in the
neutrality and reliability of public administrations.

APIs have become synonymous with facilitating the opening of large data sources
to citizens and other third parties. The Open Government imperatives have meant
that API technology has been exploited outside of the ‘IT department’,
providing access into large open data stores so that developers and their
applications and websites can more easily consume it. When a government agency
publishes an API for their data set, they open up new and innovative ways to
access the data. A developer might create a mobile or web app to display the
data intuitively or allow simple queries or automatically generate charts.

The most relevant public sector that expose government datasets is The European
Data Portal [cite 30] (EDP).

\textbf{EU Example: European Data Portal (EDP)}

The EDP provides access to 79 different catalogues, most with tens of thousands of
open datasets provided my member state governments. The same site also provides
access to over 300 use cases (services or applications) that have been developed
using the open data sets available. Some of these applications have been created
using APIs to query the EDP.

The access to the Portal is provided by a machine-readable API which enables its
users to search, create, modify and delete metadata on the portal.31 APIs are
available both via the Comprehensive Knowledge Archive Network (CKAN) [cite 32] and
SPARQL [cite 33] endpoints.

\section{APIs enable the public sector to innovate}

APIs enable new innovative service models which better engage citizens and allow
for more efficient delivery of their services. These services no longer have to
be provided directly by the agency, partners and citizen developers can use
available data to enable new solutions. Smart Cities and the vast amount of data
produced by sensors supports the development of dynamic platforms and ecosystems
providing contextualized, real-time location-based data from IoT or crowdsourcing
to business partners and startups giving them opportunities to create new services
or improve existing ones.

Transport for London have delivered successful innovation based on API use.
Although other more innovative services are coming of age in areas such as
Smart Cities, this example is one of concrete success in enhancing efficiency
and citizen service delivery.

\textbf{EU Example: TRANSPORT FOR LONDON (TfL)}

At recent European conference34, Transport for London detailed the investment that
that they had made:
\begin{itemize}
	\item 200 data elements are made available through an API to some 12,000
	developers producing some 600 apps that 40\% of Londoners use.
	\item TfL has formed partnerships with major IT players such as Apple
	(for mobile payment, rental ofbikes), Twitter (for pushing alerts out),
	a two-way data-sharing agreement with Waze (enriching the app with data
	from the road network that TfL manages while benefiting from data collected
	through Waze) and Google (enriching the maps application with real-time data).
	\item The data can be consumed under the terms of the UK Open Government
	Licence with some minimal additions for free. This is done under a statutory
	requirement as part of UK legislation.Mechanisms are in place to ensure that
	consumption remains at an acceptable level. There is one single set of data
	at the base that are both consumed by TfL for its purposes and by third	party
	developers. Developers must give attribution to TfL for the fact that their
	app includes TfL data.
	\item In terms of creation of additional economic activity, it has been
	calculated that this policy	generates GBP 100m of direct value and has enabled
	some 1,000 jobs.
	\item For data acquired by a third party, e.g. Waze data, restrictions resulting
	from the partnership agreement apply.
	\item All data made available is data that TfL collects anyway for its own
	purposes. TfL is not collecting additional data merely to make available to
	third parties.
	\item Mashing data provided by TfL with privately-held data can bring additional
	insights (e.g. "Are	there correlations between rainfall and collisions involving cyclists?”).
\end{itemize}

\section{Challenges and Considerations}

For the most part, externally facing public sector APIs involve the movement
of data that is sensitive as it often, in some way, refers to information about
a citizen. This poses a number of consistent challenges for government:
\begin{itemize}
	\item Security – APIs expose data, services, and transactions in order to
	build new services. This inherently increases the permeability of an
	organisation’s network, which can expose new vulnerabilities for
	exploitation. Therefore, APIs must be appropriately secured to ensure data
	privacy and to ensure citizen confidence in the service delivery channel.
	APIs intended for access to public data must be protected from inappropriate
	use or abuse such as denial of service. A number of security solutions exist
	such as OAuth and Certificate based authentication, which are used in
	conjunction with a wider cyber security strategy and cryptography.
	\item Regulation – APIs play a significant role in the facilitation of
	government transparency. A recent EU ruling [cite 35] makes providing
	transparency into all IT services that will be used in technology projects
	a condition for receiving government funding, and it is more than likely
	that APIs are the core technology required to support the transparency
	principle.
	\item Further regulatory considerations which must be adhered to when
	sharing data through any type of interface are the General Data Privacy
	Regulation [cite 36] (GDPR), the Payment Services Directive	(PSD2) [cite 37]
	and the Public Sector Information Directive (PSI) [cite 38].
	\item Specifications or Standards - Standards for APIs are available in
	small pockets such as the OGC [cite 39]	standard, and the developing ISO
	standard in Financial Services [cite 40]. However, many organisations are developing
	APIs based on an agreed internal specification or style guide to promote
	consistency, rather than what might normally be recognised as a de facto
	‘standard’. Each API comes with detailed documentation for consumers which
	provides clarity on the type of API	(RESTful, GRAPHQL, GRPC etc.). There
	appears to be limited appetite for further standard	development in the
	aftermath of ‘Open Government’ which is different to the impact ‘Open Banking’
	has had in the EU which precipitated the agreement of an API standard (in the
	UK initially at least) [cite 41].
\end{itemize}

The work conducted by the FIWARE Foundation (Future Internet Ware) tries to
overcome some of the challenges listed above. FIWARE will be analysed later
in the document, but in summary it is funded by a combination of EU, corporate
membership and venture capital funding and has created a scalable open source
platform used to access and manage heterogeneous context information through
open APIs [cite 42]. A standard for exchange of context information: FIWARE-NGSI
(Next Generation Service Interface) is an open standard API to be used for Smart
Cities, Smart Industry and Smart Agrifood [cite 43]. The EU has noted its success to
date, however, its success in landing a standardized API that is universally
used will be known only in time.
\begin{itemize}
	\item Business Models – In the public sector, generating income from the
	provision of data that is publically owned, and is being used for the
	public good, has not led to the charging of users who wish to consume
	or query this type of data. Examples of charging mechanisms being in place
	are	limited, one being the UK’s Ordnance Survey maps [cite 44], and KLIP (one of
	the case studies explored later in the Section dedicated to the case studies)
	which charges map requestors to have a digital map of utility services
	generated for a specific location. 
\end{itemize}

\section{Quantitative assessment of API use in the public sector}

It is hard to realistically quantify the number of public sector organisations
that are using APIs internally, but the total amount across all enterprises
and organisations is likely to run into the millions [cite 45]. Organisations
that create outwardly facing APIs to enable interaction with large data sources
are common globally. We know that they are common globally because of the
number of APIs now registered with API directories – the name given to the
many searchable catalogues of Web APIs available on the internet. In order
to ensure that APIs attract the maximum amount of developers to leverage the
data being exposed, organisations will publish their API with high-level
technical specification. Therefore, conducting an analysis of a well-recognised
directory is likely provide indicative information regarding the number of EU
public sector APIs, and the sectors and associated public services that they support.

ProgrammableWeb [cite 46] is the best known and globally recognized API directory.
Nordic APIs [cite 47] comments that it is ‘exhaustive’ and ‘comprehensive’, and is hand
curated and searchable. Therefore, as one method of obtaining quantitative,
data led insight, this study undertook a basic analysis of the almost 20,000
listed APIs (as at February 2018).

We selected the ‘Government’ category which reduced the number searched to 787.
After initial high-level analysis, our findings were that only 110 of the 787
Government category APIs advertised on the directory originated from the EU.
This may well be because of the US-based nature of ProgrammableWeb. The initial
breakdown suggested that the majority of the registered APIs were at the
National level:


TABLE

Most of the APIs provide access to open data sources for developers to use in
order to create applications for commercial sale. Others have more
democracy/citizenship based aims.

\section{Conclusion}

APIs enable cost effective data sharing through both private and public
ecosystems, which is in turn leveraged by developers to generate benefits for
the citizen, for business and for the economy. The number of APIs is continuing
to grow year on year (as demonstrated by the numbers recorded by ProgrammableWeb)
is testament to the value that they provide for the public sector across a variety
of use cases.
