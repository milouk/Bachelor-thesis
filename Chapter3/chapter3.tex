%!TEX root = ../thesis.tex
%*******************************************************************************
%****************************** Third Chapter **********************************
%*******************************************************************************
\chapter{Differences with the Private Sector}

% **************************** Define Graphics Path **************************
\ifpdf
    \graphicspath{{Chapter3/Figs/Raster/}{Chapter3/Figs/PDF/}{Chapter3/Figs/}}
\else
    \graphicspath{{Chapter3/Figs/Vector/}{Chapter3/Figs/}}
\fi

In this chapter we will compare the way the
private sector exploit APIs, and how the public sector also exploits them.

\section{API availability}

As said before, although it is hard to quantify existing APIs
as many of them are internal and unadvertised, externally available
APIs are to a certain degree registered with API directories such as ProgrammableWeb
~\citep{programmableweb_apis}, or RapidAPIs~\citep{rapid_apis}. According to a survey carried out by Deloitte~\citep{deloitte_insights} the public sector may have slower growth than the private
sector which is also believed to be slowing, or maturing in terms of underlying technology and
potential. According to
this survey, across global markets, public-sector API adoption lags and they
suggest that this is due to ongoing Open Government guidelines that
require longer time frames for organizing and executing larger scale API
transformation initiatives~\citep{deloitte_insights}. However, as discussed earlier in this
work, a huge amount of government data is being made available for
consumption by citizen and commercial developers.

\section{Private API Usage}

 Things that can be done with private APIs include ~\citep{upwork}:

\begin{itemize}
	\item Building internal apps for company use around a microservices model.
	\item Creating a shared pool of data and assets that allows teams to collaborate faster and easier.
	\item Strengthening partnerships, allowing partners to test out integrations, and streamline technical integrations.
	\item Streamline inbound and outbound marketing data collection, simplifying layered technology stacks via APIs.
	\item Building customer-facing apps with internal assets.
\end{itemize}


\section{Business Models and Disruption}

APIs have great transformative power to disrupt business, in conjunction
with other technologies such as mobile and cloud.APIs are fundamental to the digital 
disruption in the commercial space,
especially in retail, entertainment and social media~\citep{api_economy}– probably
to a far greater extent than government has been disrupted today.
\begin{itemize}
	
	\item First Utility have shown that APIs having destructive potential to alter the electric utility industry within the UK. They help users easily switch utility providers, aided by an API that enables customers to receive quotes and sign up for their service. In this way, their API is disrupting a whole industry.
	
	
	\item The impact that Netflix had on Blockbuster made possible by
	Netflix’s internal APIs, which handles two billion requests a day,
	and enables Netflix to develop and package new services for different
	platforms at speed ~\citep{netflix_api}.
	
	\item Amazon has required that all data-based communication between
	departments be done via API, naturally positioning Amazon to lead
	disruption in a world where APIs are becoming more and more ubiquitous.
	Amazon’s disruption of the book industry was closely followed by providing
	access to their cloud via APIs creating a new business now worth 160 billion.~\citep{amazon_evangelist}
	
\end{itemize}

The disruption of government may be the result of the fact that private sector,
or third sector providers can integrate with government platforms via APIs
to expose and use data to develop new and better service delivery models.
Architectures will become modular and flexible so that they can be agile and responsive to
changing demands from the ecosystem. In this way, differences with the private sector
use of APIs will converge.

\section{Making money from APIs}

APIs are becoming more and more important in terms of revenue generating activities for
business. In a recent survey of IT decision makers, Mulesoft, recently acquired
by Salesforce~\citep{mulesoft_salesforce}, a vendor of integration software found that 50\% of
large enterprises (10,000+ employees) surveyed were making more than \$10 million
a year from API initiatives~\citep{mulesoft_api_worth}. In the public sector, generating income
from the provision of data that is publicly owned, and is being used for the
public good, has rarely attracted fees. Examples of charging mechanisms being in
place are the UK’s Ordnance Survey maps and KLIP (Flanders Underground - Cable and Pipe Information Portal) which charges map requestors
to have a digital map of utility services generated for a specific location.

Governments might need to start considering financial aspects of APIs as cost pressures
become significant and might decide to adapt so as to make money from ecosystems.
The above can be achieved through a cost per API call model where, for instance transport
data is used by developers to build commercial applications. However, it is more likely for
governments to have revenue as part of a service delivery ecosystem with the private sector
that provides efficiency and cost saving. Such collaborations could be developed with cities and
local governments, insurance and health private sectors in order to create ecosystems that deliver
innovative solutions through applications.

\section{Summary}

To date, governments in order to fulfill their mission have harnessed the power of the API in order to make data more open and available to their citizens, and to themselves. The benefits range from increasing transparency, to enhanced efficiency of the existing service models.On the other hand, the private sector has harnessed APIs for a more transformative and disruptive end, developing to completely different business models, such as those which have made Netflix and
Amazon great.	
