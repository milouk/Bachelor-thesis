%!TEX root = ../thesis.tex
%*******************************************************************************
%****************************** Third Chapter **********************************
%*******************************************************************************
\chapter{Differences with the Private Sector}

% **************************** Define Graphics Path **************************
\ifpdf
    \graphicspath{{Chapter3/Figs/Raster/}{Chapter3/Figs/PDF/}{Chapter3/Figs/}}
\else
    \graphicspath{{Chapter3/Figs/Vector/}{Chapter3/Figs/}}
\fi

In this section of the study we will explore the differences between how the
private sector exploit APIs, and how the public sector also exploits them.

\section{API availability}

As mentioned before although it is hard to quantify the number of APIs that
are in existence
due to many of them being internal unadvertised APIs, externally available
APIs are to some degree tracked by API directories such as ProgrammableWeb
~\citep{programmableweb_apis}, or RapidAPIs~\citep{rapid_apis}. A relatively recent survey by Deloitte~\citep{deloitte_insights}
indicates that the public sector may have slower growth than the private
sector which is also deemed to be slowing, or maturing. According to
Deloitte, across global markets, public-sector API adoption lags and they
suggest that this may be due to ongoing Open Government guidelines that
mandate longer time frames for organizing and executing larger scale API
transformation initiatives~\citep{deloitte_insights}. However, as explored earlier in this
paper, a huge amount of government data is being made available for
exploitation by citizen developers and commercial developers alike.

\section{Business Models and Disruption}

APIs have great transformative powers to disrupt business, when coupled
with other technologies such as the powerful forces of Mobile and Cloud.
The API is integral to the digital disruption in the commercial space,
especially in retail, entertainment and social media~\citep{api_economy}– probably
to a far greater extent than government has been disrupted today. Some
of the world’s biggest brands have been significantly disrupted, or taken
out of business by a new breed of companies that leverage technology to
open up different ways of providing much sought-after services.
\begin{itemize}
	\item The impact that Netflix had on Blockbuster made possible by
	Netflix’s internal API, which handles two billion requests a day,
	and enables Netflix to develop and package new services for different
	platforms at speed.
	
	\item Amazon has required that all data-based communication between
	departments be done via API, naturally positioning Amazon to lead
	disruption in a world where APIs are becoming more and more ubiquitous.
	Amazon’s disruption of the book industry was closely followed by providing
	access to their cloud via APIs creating a new business now worth 160 billion.
	
	\item Dun and Bradstreet (D\&B) as another example of APIs disrupting
	traditional business. The long established credit approval company has
	innovated with their API, enabling D\&B lookups to be performed from
	within third-party apps, or within SaaS services such as SalesForce.com
	leveraged APIs, creating a new revenue channel and disrupting the industry.
	
	\item First Utility, have demonstrated that APIs having destructive
	potential to alter the electric utility	industry within the UK. They
	help users easily switch utility providers, aided by an API that enables
	customers to receive quotes and sign up for their service. In this way,
	their API is disrupting a whole industry.
\end{itemize}

The disruption of government may be driven by the ease in which private sector,
or even third sector providers can integrate with government platforms via APIs
to share and use data to drive new and improved service delivery models.
Gartner’s series of recent papers on the state of government in 2030 predicts
that governments will relinquish service delivery by empowering the ecosystem,
through intelligence and innovation, to improve citizen services. Architectures
will become modular and flexible so that they can be agile and responsive to
changing demands from the ecosystem. Thus, differences with the private sector
use of APIs will converge.

\section{Making money from APIs}

APIs are an increasingly important part of revenue generating activities for
business. In a recent survey of IT decision makers, Mulesoft, recently acquired
by Salesforce~\citep{mulesoft_salesforce}, a vendor of integration software found that 50\% of
large enterprises (10,000+ employees) surveyed were making more than \$10 million
a year from API initiatives~\citep{mulesoft_api_worth}. In the public sector, generating income
from the provision of data that is publicly owned, and is being used for the
public good, has rarely attracted fees. Examples of charging mechanisms being in
place are the UK’s Ordnance Survey maps and KLIP  which charges map requestors
to have a digital map of utility services generated for a specific location.

Governments may need to become more financially driven/savvy as cost pressures
become significant, and may choose to adapt to make revenue streams from ecosystems.
This may manifest itself in a cost per API call model where, for example, transport
data is being used by developers to create commercial applications. However, it is
more likely to be part of a service delivery ecosystem with the private sector that
provides efficiency and cost saving. For example, cities and local governments are
predicted to collaborate with the automotive, insurance and health sectors to create ecosystems powering applications that deliver
innovative solutions.

\section{Conclusion}

To date, governments has harnessed the power of the API to make data more open and
available to their citizens, and to themselves. The benefits range from increasing transparency, to enhanced efficiency of the existing service models. The private
sector has harnessed APIs for a more transformative and disruptive end, giving rise
to completely different business models, such as those which have made Netflix and
Amazon great. The next section will deal with what the public sector may do in the
future to disrupt itself in the face of increasing citizen demand, and cost pressures.	
