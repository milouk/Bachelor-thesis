%!TEX root = ../thesis.tex
%*******************************************************************************
%****************************** Fourth Chapter **********************************
%*******************************************************************************
\chapter{The future trends for API use in the Public Sector}

% **************************** Define Graphics Path **************************
\ifpdf
    \graphicspath{{Chapter4/Figs/Raster/}{Chapter4/Figs/PDF/}{Chapter4/Figs/}}
\else
    \graphicspath{{Chapter4/Figs/Vector/}{Chapter4/Figs/}}
\fi

In this section, we will identify current thinking on how the use of APIs may
evolve over the next 3 to 5 years.
\begin{itemize}
	\item Growth Rate - There is some evidence that the growth of APIs has slowed
	to some degree [cite 63]. However, although the number of APIs may not be
	growing at the rate that was predicted a few years ago, their use and the
	ecosystems that they support continue to thrive.
	\item Digital Government Platform growth requires APIs - Predictions on the
	future trends in Digital Government from research companies such Forrester
	and Gartner indicate that Digital Government Platforms (interoperable,
	horizontal microservices that are orchestrated by RPA (Robotic Process
	Automation) software will become more prevalent in the 3-5 year window [cite 64].
	Digital Government Platforms require APIs as the integration mechanism to move
	data between component systems and therefore governments will continue to
	invest in switching from a service-oriented architecture (SOA) to a modular
	architecture (MASA) utilising APIs and micro-services.
	\item Government will invest in Intelligent Things requiring APIs – It is
	likely that governments will continue to increase investments in intelligent
	things, across many domains — from defence,	policing, waste management, health, agriculture and smart communities [cite 65] to enhance service delivery quality,
	and	efficiency. Sensor and video networks, intelligent drones, fleets of automated
	vehicles, and robotic devices will become core to government service delivery
	capability and serve as a real-time data source for government, using APIs to
	transfer data among IT systems and layers. It is anticipated that the next
	progression will see the environment composed of many physical things with both
	sensor and computation capabilities, which make the technology direction
	pervasive and invisible [cite 66].	Applications will be capable of
	communication, cooperation, and negotiation with each other. Unlike general
	applications, agents will be designed with goals to be fulfilled on behalf
	of its users. That is, agents will take necessary actions efficiently towards
	its environment over the P2P protocol. For example, an agent can be designed
	to read a patient’s biometrics from a patients wearable sensor devices and
	adjust thermostats to heat or cool a patient’s room appropriately. In this way,
	the new platform is not limited to a certain set of devices, and it opens many possibilities over the P2P protocol to produce novel (multi-Agent) applications
	that enrich the idea of ubiquitous computing [cite 67].
	\item APIs as products – APIs are products, and as such should have a
	product lifecycle from conception and improvement through to retirement.
	Government IT departments will continue to move away from APIs just being
	a technology to implement and forget. Given growing ecosystems dependent
	on APIs, communicating to third-parties, monitoring usage, and removing at
	an acceptable time (i.e. versioning) will be important.
	\item API Standards – The cost savings that can be realised by not having
	to redesign an API due to its ‘drag and drop’ portability seem compelling.
	The ability for applications and data sources to be able to link without
	the need for a bespoke API takes us one step closer to the ubiquitous platform
	of unbounded data. However, for many, knowing the specification of an API,
	and then getting to know the specific nuances of the API via the developer
	portal of an API provider seems to be a	necessary, and perhaps keeps the
	developer community in work.
	\item Citizen developers and Open APIs - Open APIs make it very easy for
	citizens to make use of	open data, or improve existing applications which
	leverage it. Hackathons will become more widespread as a way for the public
	sector to engage with citizens, helping member states to meet the aims that
	they have of conducting significant user research prior to releasing any
	citizen facing services or data [cite 68] changes could potentially come
	faster than if we were to wait for the vendor to implement them. This process
	is very similar to open-source software, which is widely used and very
	helpful for developers.
\end{itemize}

\section{Conclusion}

The future of digital government seems deeply linked to the use of the APIs as
enablers. As the technological demands of digital government move forward, it
appears that APIs are well positioned to keep pace, and provide the access points
needed to enable fast and secure data sharing to support government’s needs from
law and order, to healthcare and the environment. As with all aspects of technology,
the use and development of APIs will evolve over time.