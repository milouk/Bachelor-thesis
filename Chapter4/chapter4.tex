%!TEX root = ../thesis.tex
%*******************************************************************************
%****************************** Fourth Chapter **********************************
%*******************************************************************************
\chapter{The future trends for API use in the Public Sector}

% **************************** Define Graphics Path **************************
\ifpdf
    \graphicspath{{Chapter4/Figs/Raster/}{Chapter4/Figs/PDF/}{Chapter4/Figs/}}
\else
    \graphicspath{{Chapter4/Figs/Vector/}{Chapter4/Figs/}}
\fi

In this section, we will point out what the future trends of APIs in the public sector may be
in a time span of 5 years.

\begin{itemize}
	\item \textbf{Growth Rate} - There is some clue that the growth of APIs has decelerate
	to a certain extent~\citep{api_economy}. However, although the number of APIs may not be
	growing at the pace that was anticipated a few years ago, their use and the
	ecosystems that they support continue to grow.
	
	\item \textbf{Digital Government Platform growth requires APIs} - Predictions on the
	future trends in Digital Government from research companies such Forrester
	suggest that Digital Government Platforms software will become more
	common in the next 3-5 years~\citep{gov_transformation}.
	Digital Government Platforms require APIs as the integration mechanism to transfer
	data between component systems and therefore governments will continue to
	invest in switching from a service-oriented architecture (SOA) to a modular
	one (MASA) exploiting APIs and micro-services.
	
	\item \textbf{Government will invest in Intelligent Things requiring APIs} – It is
	expected that governments will go on increasing investments in intelligent
	solutions, across many sectors — from defense,	policing, waste management, health, agriculture and smart communities~\citep{trends_2017} in order to enhance service delivery quality, and efficiency. Sensor and video networks, intelligent drones, fleets of automated
	vehicles, and robotic devices will become fundamental to government service delivery
	capability and serve as a real-time data source for government, using APIs to
	move data among IT systems and layers. It is expected that the next
	progression will see the environment composed of many physical things with both
	sensor and computation capabilities, will make the technology direction
	pervasive and invisible~\citep{dzone_iot}. Applications will be capable of
	communication, cooperation, and negotiation with each other. Unlike general
	applications, agents will be designed with targets to be fulfilled on behalf
	of its users. That is, agents will take appropriate actions efficiently towards
	its environment over the P2P protocol. For example, an agent can be designed
	to read a patient’s biometrics from a patients wearable sensor devices such as
	a smartwatch and adjust thermostats to heat or cool a patient’s room accordingly.
	In this way,the new platform is not limited to a certain set of devices, and it opens many possibilities over the P2P protocol to produce novel (multi-Agent) applications
	that enrich the idea of ubiquitous computing~\citep{ubiquitous_computing}.
	
	\item \textbf{APIs as products} – APIs are products and as such should have a
	product lifecycle from conception and development through to withdrawal.
	In a similar manner given the fact that API is a technology to implement and forget 
	Government IT departments will need to communicate with third parties that depend on them,
	monitor their usage and withdraw them when needed (i.e versioning).
	
	\item \textbf{API Standards} – Not having to redesign an API due to its modular
	capabilities is cost saving and compelling for the public sector.
	The ability for applications and data sources to be able to connect without
	the need for a bespoke API takes us one step closer to the ubiquitous platform
	of open data. However, it is of utmost importance for the developer community, to know the standards of an API,
	and then getting to know the specifications of the API usually through the developer
	portal of an API provider.
	
	\item \textbf{Citizen developers and Open APIs} - With Open APIs
	citizens can easily make use of	open data, or improve already existing applications which
	leverage it. Hackathons will become more widespread as a way for the public
	sector to interact with people. This helps member states in their aim to conduct significant user research before releasing any
	citizen facing services. Moreover data changes could potentially come
	faster than if we were to wait for the vendor to implement them. This process
	is very similar to open-source software, which is widely used and very
	helpful for developers~\citep{digital_first}.
	
	\item \textbf{AI and ML-based APIs} - Using predictive analytics APIs to combine big data, embedded, visual, spatial/location, text, web, network and mobile information has evolved to include natural language processing especially within context per Business Intelligence trends~\citep{bisurvey}.
	New sources of real-time data help with trend detection for faster responses to the intelligence. For example, Netflix now offers interactive content that simulates the Choose Your Own Adventure format most notable in a  “Black Mirror” special ~\citep{bms}. Another example is how Axway partnered with Elastic Beam to leverage an AI API ~\citep{axway}.
\end{itemize}

\section{Conclusion}

The future of digital government seems deeply linked to the use of the APIs as
enablers. As the technological demands of digital government move forward, it
appears that APIs are well positioned to keep pace, and provide the access points
needed to enable fast and secure data sharing to support government’s needs from
law and order, to healthcare and the environment. As with all aspects of technology,
the use and development of APIs will evolve over time.