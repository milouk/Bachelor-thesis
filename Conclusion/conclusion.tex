%!TEX root = ../thesis.tex
%*******************************************************************************
%****************************** Conclusion Chapter **********************************
%*******************************************************************************
\chapter{Conclusion}

% **************************** Define Graphics Path **************************
\ifpdf
\graphicspath{{Conclusion/Figs/Raster/}{Conclusion/Figs/PDF/}{Conclusion/Figs/}}
\else
\graphicspath{{Conclusion/Figs/Vector/}{Conclusion/Figs/}}
\fi

This study set out to explore the API landscape in the EU public sector. The purpose of the study has been
to identify areas in the ability of APIs to assist member states with enabling their digital transformation.
Areas of specific focus include aspects such as cross-border interoperability between member states, and
the opportunity for the EU to become involved in developing or advocating API standards. To deliver the
insight required both desk based research and structured interviews with public sector organisations that
have developed successful APIs were carried out.
The report provides a useful baseline overview of APIs, considering what they are used for, the different
types of API that can be leveraged, and the API standards that exist. A glossary of terms and API types in
the appendices provide further resources for the target audience. The report then goes on to consider how
APIs are used in the public sector. The findings showed that APIs are used by the public sector to help
them achieve their goals in four main ways:
1. Enabling ecosystems.
2. Overcoming complex integration of large systems.
3. Supporting Open Government initiatives.
4. Enabling innovation and economic growth.
The use of APIs is not without its challenges however. This study highlighted security and enhanced EU
regulation around privacy as considerations for API owners. An API is another gateway into a computer
network and associated data, and requires the security features and ongoing maintenance that such an
interface deserves.
The lack of standards (except in the geospatial/mapping space where the OGC has many) was also
considered, both in the desk based research and the interviews conducted. In summary, the lack of
standards does in some way hinder interoperability both internally and externally to government agencies. It
is forcing organisations to develop their own set of guidelines to ensure alignment, and this is something
that the UK Government have recently released to all API developers75. However, the use of API gateways,
and the predominance of RESTful architectures is in some way diluting the pressure for a standard. When
case study interviewees were questioned on the potential role of the EU in developing a standard they saw
the benefit, but were also cautious. Respondents were keen that anything that the EU developed, or
advocated was ‘lightweight’ and did not try to be all encompassing and theoretical.
The study also considers the relationship between APIs and location data. It concludes that APIs provide
access to various aspects of location data, and assist in retrieving a variety of data points which when
‘mashed’ with other contextual data can provide a powerful tool for the state, or the citizen. Many use cases are in operation in spheres such as weather, emergency resilience, Smart Cities, and Gazetteers to name
only a few.
Differences with the private sector were also considered. The report found that to date, government has (in
the main) harnessed the power of the API to make data more open and available to their citizens, and to
themselves. The benefits range from increasing transparency, to enhanced efficiency of the existing service
models. The private sector has harnessed APIs for a more transformative and disruptive end, giving rise to
completely different business models, such as those which have made Netflix and Amazon leaders in their
field.
Our research also considered the future of government, which will be to some extent built on the API as a
key enabler. As the demands of government move forward, it appears that APIs are well positioned to keep
pace, and provide the access points needed to enable fast and secure data sharing to support
government’s needs from law and order, healthcare and the environment.
Our case studies provided many interesting insights into successful API adoption. Many noted the
importance of the use of Agile methods, and the impact of legislation/policy to stimulate uptake and
development of APIs which have given rise to substantial benefits. The benefits were probably the most
revealing aspect to this part of the study, providing compelling evidence that solutions with APIs at their
core such as Estonia X-Road, Amsterdam City Data Web API and Danish Address Web API are providing
substantial returns on investment, in the case of X-Road this amounts to 800 person/years being saved
every year. Not only does this give rise to more efficient public services, it helps government keep pace with
citizens expectations.
Finally, in line with its purpose, the study suggests a number of further topics be considered. The most
significant is in relation to the development of an EU API standard. This is clearly an area that can deliver
benefit, and has support (based on our limited study), however, there are some key design principles that
would need to be explored with a wider audience as a next step – and this audience must include API
consumers and providers, not be developed in isolation or academically. Other areas for consideration are
regarding the economic stimulation provided by APIs, and the way in which APIs will play a role in the future
of government enabling wider ecosystems incorporating the private sector, and the exploitation of disruptive
tools such as AI and Robotics

