%!TEX root = ../thesis.tex
%*******************************************************************************
%****************************** Conclusion Chapter **********************************
%*******************************************************************************
\chapter{Conclusion}

% **************************** Define Graphics Path **************************
\ifpdf
\graphicspath{{Conclusion/Figs/Raster/}{Conclusion/Figs/PDF/}{Conclusion/Figs/}}
\else
\graphicspath{{Conclusion/Figs/Vector/}{Conclusion/Figs/}}
\fi

This study set out to explore the API landscape in the EU public sector. The
purpose of the study has been to identify areas where APIs are enablers of governments
digital transformation. Areas of specific focus
include aspects such as sources of open data, differences between APIs in the private sector
and the future trends of APIs.

The report provides a useful baseline overview of APIs, considering what they are
used for, the different types of API that can be leveraged, and the API standards
that exist. A glossary of terms and API types in the appendices provide further
resources for the target audience. The report then goes on to consider how APIs
are used in the public sector. The findings showed that APIs are used by the public
sector to help them achieve their goals in four main ways:
\begin{itemize}
	\item Enabling ecosystems.
	\item Overcoming complex integration of large systems.
	\item Supporting open government initiatives.
	\item Enabling innovation and economic growth.
\end{itemize}

The use of APIs has its challenges too. This study highlighted security
and enhanced EU regulation around privacy as considerations for API owners. An API
is another gateway into a computer network and associated data, and requires the security features and ongoing maintenance that such an interface deserves.

The lack of standards does in some way hinder interoperability both internally and
externally to government agencies. It is forcing organizations to develop their own set of guidelines to ensure alignment, and this is something that the UK
Government have recently released to all API developers~\citep{gov_uk_api}. However, the use
of API gateways, and the predominance of RESTful architectures is in some way diluting
the pressure for a standard.

Differences with the private sector were also considered. The report found that to date, government has harnessed the power of the API to make data more open and available to their citizens, and to themselves. The benefits range from increasing transparency, to enhanced efficiency of the existing service models. The private sector
has harnessed APIs for a more transformative and disruptive end, giving rise to
completely different business models, such as those which have made Netflix and Amazon leaders in their field.

Our research also considered the future of government, which will be to some extent
built on the API as a key enabler. As the demands of government move forward, it
appears that APIs are well positioned to keep pace, and provide the access points
needed to enable fast and secure data sharing to support government’s needs from
law and order, healthcare and the environment.

Our study provides brief examples of governemnt solutions with APIs at their core such as 
Estonia's X-Road, UK's TfL and Greece's Digital Solemn Declaration/ Authorization issuing system.

We truly believe that our study has given some useful insight in how APIs as a technology
can contribute to Governments digital transformation.

