%!TEX root = ../thesis.tex
%*******************************************************************************
%******************************** Internship Report ****************************
%*******************************************************************************

\chapter{Report}

\section{Company Description}
GRNET S.A. provides Internet connectivity,
high-quality e-Infrastructures and advanced services to the Greek Educational,
Academic and Research community,
aiming at minimizing the digital divide and
at ensuring equal participation of its members
in the global Society of Knowledge.
Additionally, GRNET develops digital applications that
ensure resource optimization for the Greek State,
modernize public functional structures and procedures,
and introduce new models of cooperation between public bodies,
research and education communities, citizens and businesses.\par

GRNET provides advanced services to the following sectors:
Education, Research, Health, Culture.\par

GRNET operates under the auspices of
the Greek Secretariat for Research and Technology /
Ministry of Education, Research and Religious Affairs.

\subsubsection{ADVANCED E-INFRASTRUCTURES AND INNOVATIVE SERVICES}
GRNET backbone interconnects more than 100 institutions including
all universities and technological institutions,
research centers, public hospitals, museums and libraries,
as well as the Greek School Network,
with speeds up to 26*10 Gbps through its high-speed,
high-capacity infrastructure of long-term leased fiber
that spans across the entire country.\par

GRNET is present in global networking for research and education,
representing Greece in GÉANT.
GÉANT is Europe’s leading collaboration on network
and related infrastructure and services
for the benefit of research and education,
contributing to Europe’s economic growth and competitiveness.

\subsubsection{LARGE SCALE COMPUTING SERVICES FOR THE RESEARCHERS (CLOUD COMPUTING)}
GRNET offers innovative Cloud Computing services
that are available via the Infrastracture as a Service model,
under the brand name “\textasciitilde okeanos”.
By using “\textasciitilde okeanos”,
any academic user can create a multi-layer virtual infrastructure
and instantiate virtual computing machines,
local networks to interconnect them,
and a reliable storage space within seconds.
Thousands of academic users have already utilized virtual machines
in the course of their research, experimental, educational or other activities.
The Cloud Computing infrastructure and services of GRNET
have been made available to the pan-European R\&E community
via the “okeanos-global” service.

\subsubsection{HELLENIC HIGH PERFORMANCE COMPUTING INFRASTRUCTURE}
GRNET “ARIS” national high-performance computing infrastructure
provides state-of-the-art supercomputing capabilities
to the Greek scientists.
The system went into pilot operational phase in June 2015
and it is available for productive use
to all researchers and scientists across
Greek universities, technological institutions and research centers.
The system enables the implementation of scientific
and technical large-scale applications,
with GRNET guaranteeing its smooth operation
offering comprehensive end-user support.\par

ARIS is based on IBM’s NeXtScale platform,
incorporating the Intel ® Xeon ® E5 v2 processors,
(Ivy Bridge) and it provides computational power
that reaches 170TFlops (trillion floating point operations per second).
With a total of 426 compute nodes,
it offers more than 8500 processor cores (CPU cores)
interconnected through FDR Infiniband network,
a technology offering very low latency and high bandwidth.
In addition, the system will offer about
1 Petabyte (quadrillion bytes) of raw storage,
based on the IBM General Parallel File System (GPFS).
The system software allows developing
and running scientific applications
and provides several pre-installed compilers,
scientific libraries and popular scientific application suites.

\subsubsection{ENHANCING THE USE OF ICT AND ACCESS TO DIGITAL CONTENT}
GRNET coordinates a series of initiatives
aimed at creating e-infrastructures and services
that can facilitate organizing, describing and promoting
digital content of educational, research, geospatial,
and environmental as well as cultural topics.
These actions contribute to the vision
of creating a virtual horizontal infrastructure of digital repositories,
which is available from universities, research centers, museums, libraries
and other institutions in the country and Europe and
facilitates the preservation, sharing and exploitation
of digital content by businesses and the society.
The ultimate goal is to enhance the use of digital content
and services from researchers, teachers, staff of public bodies and SMEs,
as well as other types of communities involved in
the production, processing, and use of digital knowledge.

\subsubsection{ENVIRONMENTAL POLICY}
GRNET implements green technologies
in its networking and computing infrastructure,
in an attempt to reduce its yearly greenhouse gas emissions’ footprint.
In order to achieve that,
environmental regulation, laws and codes of practice
are highly regarded when assessing standards of environmental performance.\par

To this respect, GRNET’s environmental policy is based
on the following lines of action:

\begin{itemize}
	\item upgrades of networking \& computational infrastructure
	with energy efficient equipment,
	\item deployment of an energy consumption monitoring infrastructure
	for real time measurements in
	the network Points of Presence (PoPs) and Data Centers,
	\item participation in research activities for the design of
	energy-aware mechanisms in the operation and control of the network,
	\item improvement of energy efficiency in GRNET data centers
	(low Power Usage Effectiveness – PUE values) through
	the application of innovative energy-aware techniques,
	\item increase of environmental awareness within
	the Greek research and academic community
	through the dissemination of “green” best practices,
	\item increase in the use of videoconferencing tools
	to achieve commute trip reduction,
	\item minimization of environmental pollution through the reduction,
	reuse or recycling of materials.
\end{itemize}

\subsubsection{GR-IX}
GRNET operates GR-IX, the Greek Internet Exchange.
GR-IX constitutes an important national infrastructure
as it interconnects significant players of the Greek Internet,
such as Internet Service Providers, Content providers etc.,
and facilitates the exchange of IP traffic among them.

The goal of GR-IX is to improve the connectivity,
quality and speed of the Greek Internet and,
at the same time, reduce the cost of accessing it.

\section{Internship Goal}
The general goal of this internship, is gaining experience in Software Engineering through the exposure to the complete lifecycle of Software Development, understanding how each step impacts the rest. The main focus is towards the coding aspect of SDLC, writing reusable, clean and well documented software.

\section{Final Report Goal}
The goal of the final report , is to give an overview of the things that i have interacted with, throughout my internship, analyzing thoroughly both the deliverables and the theory behind the technologies used.


\section{Basic Characteristics}\label{ch:basic_characteristics}

\subsection{Organic Structure}
The Organic Structure consists of the Management Bodies,
as provided for in Presidential Decree 29/1998,
as amended and in force, the Support Structures and the Directorates.
The Board of Directors of GRNET SA,
which determines the direction in which the Company moves and takes all final decisions, is:
The General Assembly (GA), the Board of Directors (SA),
the Chairman and Chief Executive Officer and the Deputy Chairman BoD.
The organizational structure of GRNET SA is illustrated in Figure 1.

\subsection{Board of Directors}
The Board of Directors is the highest administrative body of GRNET SA.
and is competent to decide on any matter relating to the administration and the achievement of the Company's aims,
except for the matters relating to the exclusive functions of the General Assembly.
The Board of Directors mainly shapes the Company's strategy and development policy,
secures the main sources of funding of GRNET SA while supervising and controlling the management of its assets.
It also cultivates and promotes the national and European dimension of the Company.
The BoD has eight members and his term of office is four years.
The Chairman is a member of the BoD, as well as the Company's supreme executive.
It carries out the scientific supervision of the functions of GRNET SA,
presides over all its Directorates and Services and directs its work, while proposing to the Board on policy issues.

\subsection{Directorate of Administrative and Financial Management}
The Directorate of Administrative Functions and Financial Management assists the Board of Directors and the Chairman
in their work and supports the Company in relation to the following:
It supports Administrative, Operational and Secretarial Management of the Company,
its Management Bodies, Human Resources and its Projects. Technologically supports the Company's Management Bodies,
and Human Resources on Internal ICT Infrastructure.
It processes the Company's supplies.
He deals with Human Resource Management issues.
It supports the Company in managing all funded projects.
It carries out the accounting management and financial monitoring of the Company and the financed Projects.

\subsection{Technology Development Division}
The Technology Development Division is responsible for designing the long-term development of electronic infrastructures and services,
either autonomously or through participation in infrastructure projects (national or international),
in line with the company's strategic objectives and in co-operation with other company's departments.
It also has the responsibility to implement the technology development plan
as well as to cooperate with the GRNET organizations for the timely analysis
and more effective satisfaction of their needs.
It is also responsible for the development of systems for the automated provision
and management of services as well as for the study, search and execution of European and International Projects.

\subsection{Directorate of Electronic Infrastructure}
The Directorate of Electronic Infrastructure has the responsibility to operate, upgrades, install, modify
and supervise the performance of the productive infrastructure and services,
maintenance and costing of its operation, the management of the GRNET support center and infrastructure support teams,
contacts with contractors, suppliers and subcontractors.

\subsection{Information Services and Application Development Division}
The Information Services and Application Development Division is responsible for the full life cycle of the projects
and applications deployed internally in the company, from the planning phase to the maintenance process.
It also owns the information services that the company has to the users of the wider educational and research community.
It helps define the company's strategy in the field of ICT and its implementation.
It also participates in the submission of selected proposals and in the execution of national and European projects.

\subsubsection{The Information Services and Application Development Division}
The Information Services and Application Development Division is dealing with:

\begin{itemize}
	
	\item Developing and managing the entire software tool life cycle to implement and manage e-services
	and infrastructures that ensure technological homogeneity, minimize development and maintenance costs,
	and exploit synergies between the technical requirements of various software products.
	
	\item The design and maintenance of the software for the operation and management of the company's basic cloud computing services.
	
	\item The submission of proposals for innovative technological applications and services to national and Community authorities.
	
	\item Evaluation, feasibility study and technical assessment of future application development.
	
	\item  The promotion of open technologies and the company's contribution to open source projects.
	
	\item Dissemination of best technology practices and software development methodologies.
	
	\item Designing e-Services to support the educational and research community.
	
	\item Supervising the implementation, support and good functioning of the services.
	
	\item Coordination of individual user service groups for information services and allocation of management staff to individual services.
	
	\item Optimizing the business operation of large online information systems developed within the services.
	
	\item Managing the interoperability of these services and their respective services
	
	registers created in the context of their operation.
	
	\item Introducing good practices for the administration, management and implementation of
	
	information services.
	
	\item The preparation and execution of projects (NSRF, PIP, etc.), under which
	
	electronic services are being developed.
	
	\item Implementing Related Projects.
\end{itemize}

The Division consists of the following departments:
\begin{itemize}
	
	\item Design and Application Development Department
	
	\item Department of Operation and Maintenance of Information Services
\end{itemize}

\subsection{The Project - DILOSI}
\subsubsection{Digital Solemn Declaration}
fdgdfgfdgfdgfgdfg


\subsection{My Role}
sdkfdskfdfdsf

\textbf{Requirements}
\begin{itemize}
	\item Ability to produce high quality, maintainable and reusable code using Python.
	\item Ability to develop and consume RESTful web services.
	\item Good knowledge of UNIX based systems.
	\item Decent familiarity with Git.
	\item Ability to communicate ideas, meet deadlines and be an active team member.
\end{itemize}