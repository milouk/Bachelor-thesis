%!TEX root = ../thesis.tex
%*******************************************************************************
%******************************** Internship Report ****************************
%*******************************************************************************

\chapter{Report}

The goal of this report is to give an overview of the things that I have interacted with, throughout my internship, analyzing thoroughly both the deliverables and the theory behind the technologies used.

\section{Company Description}
GRNET S.A. provides Internet connectivity,
high-quality e-Infrastructures and advanced services to the Greek Educational,
Academic and Research community,
aiming at minimizing the digital divide and
at ensuring equal participation of its members
in the global society of knowledge.
Additionally, GRNET develops digital applications that
ensure resource optimization for the Greek state,
modernize public functional structures and procedures,
and introduce new models of cooperation between public bodies,
research and education communities, citizens and businesses.

GRNET provides advanced services to the following sectors:
Education, Research, Health, Culture.

GRNET operates under the auspices of
the Greek Secretariat for Research and Technology /
Ministry of Education, Research and Religious Affairs.

\subsubsection{Advanced E-Infrastructures and Innovative Services}
The backbone interconnects more than 100 institutions including
all universities and technological institutions,
research centers, public hospitals, museums and libraries,
as well as the Greek School Network,
with speeds up to 26*10 Gbps through its high-speed,
high-capacity infrastructure of long-term leased fiber
that spans across the entire country.\par

GRNET is present in global networking for research and education,
representing Greece in GÉANT.
GÉANT is Europe’s leading collaboration on network
and related infrastructure and services
for the benefit of research and education,
contributing to Europe’s economic growth and competitiveness.

\subsubsection{Large Scale Computing Services for the Researchers - Cloud Computing}
GRNET offers innovative Cloud Computing services
that are available via the Infrastracture as a Service model,
under the brand name “\textasciitilde okeanos”.
By using “\textasciitilde okeanos”,
any academic user can create a multi-layer virtual infrastructure
and instantiate virtual computing machines,
local networks to interconnect them,
and a reliable storage space within seconds.
Thousands of academic users have already utilized virtual machines
in the course of their research, experimental, educational or other activities.
The Cloud Computing infrastructure and services of GRNET
have been made available to the pan-European R\&E community
via the “okeanos-global” service.

\subsubsection{Hellenic High Performance Computing Infrastructure}
The “ARIS” national high-performance computing infrastructure
provides state-of-the-art supercomputing capabilities
to the Greek scientists.
The system went into pilot operational phase in June 2015
and it is available for productive use
to all researchers and scientists across
Greek universities, technological institutions and research centers.
The system enables the implementation of scientific
and technical large-scale applications,
with GRNET guaranteeing its smooth operation
offering comprehensive end-user support.

ARIS is based on IBM’s NeXtScale platform,
incorporating the Intel ® Xeon ® E5 v2 processors,
(Ivy Bridge) and it provides computational power
that reaches 170TFlops (trillion floating point operations per second).
With a total of 426 compute nodes,
it offers more than 8500 processor cores (CPU cores)
interconnected through FDR Infiniband network,
a technology offering very low latency and high bandwidth.
In addition, the system will offer about
1 Petabyte (quadrillion bytes) of raw storage,
based on the IBM General Parallel File System (GPFS).
The system software allows developing
and running scientific applications
and provides several pre-installed compilers,
scientific libraries and popular scientific application suites.

\subsubsection{Enhancing the Use of ICT and Access to Digital Content}
GRNET coordinates a series of initiatives
aimed at creating e-infrastructures and services
that can facilitate organizing, describing and promoting
digital content of educational, research, geospatial,
and environmental as well as cultural topics.
These actions contribute to the vision
of creating a virtual horizontal infrastructure of digital repositories,
which is available from universities, research centers, museums, libraries
and other institutions in the country and Europe and
facilitates the preservation, sharing and exploitation
of digital content by businesses and the society.
The ultimate goal is to enhance the use of digital content
and services from researchers, teachers, staff of public bodies and SMEs,
as well as other types of communities involved in
the production, processing, and use of digital knowledge.

\subsubsection{Environmental Policies}
GRNET implements green technologies
in its networking and computing infrastructure,
in an attempt to reduce its yearly greenhouse gas emissions’ footprint.
In order to achieve that,
environmental regulation, laws and codes of practice
are highly regarded when assessing standards of environmental performance.

To this respect, GRNET’s environmental policy is based
on the following lines of action:

\begin{itemize}
	\item upgrades of networking \& computational infrastructure
	with energy efficient equipment,
	\item deployment of an energy consumption monitoring infrastructure
	for real time measurements in
	the network Points of Presence (PoPs) and Data Centers,
	\item participation in research activities for the design of
	energy-aware mechanisms in the operation and control of the network,
	\item improvement of energy efficiency in GRNET data centers
	(low Power Usage Effectiveness – PUE values) through
	the application of innovative energy-aware techniques,
	\item increase of environmental awareness within
	the Greek research and academic community
	through the dissemination of “green” best practices,
	\item increase in the use of videoconferencing tools
	to achieve commute trip reduction,
	\item minimization of environmental pollution through the reduction,
	reuse or recycling of materials.
\end{itemize}

\subsubsection{GR-IX}
GRNET operates GR-IX, the Greek Internet Exchange.
GR-IX constitutes an important national infrastructure
as it interconnects significant players of the Greek Internet,
such as Internet Service Providers, Content providers etc.,
and facilitates the exchange of IP traffic among them.

The goal of GR-IX is to improve the connectivity,
quality and speed of the Greek Internet and,
at the same time, reduce the cost of accessing it.


\subsection{Internship Goal}
The general goal of this internship, is gaining experience in Software Engineering through the exposure to the complete lifecycle of Software Development, understanding how each step impacts the rest. The main focus is towards the coding aspect of System Development Life Cycle (SDLC), writing reusable, clean and well documented software.

\section{Internship Basic Characteristics}\label{ch:basic_characteristics}

\subsection{Organic Structure}
The Organic Structure consists of the Management Bodies,
as provided for in Presidential Decree 29/1998,
as amended and in force, the Support Structures and the Directorates.
The Board of Directors of GRNET SA,
which determines the direction in which the Company moves and takes all final decisions, is:
The General Assembly (GA), the Board of Directors (SA),
the Chairman and Chief Executive Officer and the Deputy Chairman BoD.
The organizational structure of GRNET SA.

\subsection{Board of Directors}
The Board of Directors is the highest administrative body of GRNET SA.
and is competent to decide on any matter relating to the administration and the achievement of the Company's aims,
except for the matters relating to the exclusive functions of the General Assembly.
The Board of Directors mainly shapes the Company's strategy and development policy,
secures the main sources of funding of GRNET SA while supervising and controlling the management of its assets.
It also cultivates and promotes the national and European dimension of the Company.
The BoD has eight members and his term of office is four years.
The Chairman is a member of the BoD, as well as the Company's supreme executive.
It carries out the scientific supervision of the functions of GRNET SA,
presides over all its Directorates and Services and directs its work, while proposing to the Board on policy issues.

\subsection{Directorate of Administrative and Financial Management}
The Directorate of Administrative Functions and Financial Management assists the Board of Directors and the Chairman
in their work and supports the Company in relation to the following:
It supports Administrative, Operational and Secretarial Management of the Company,
its Management Bodies, Human Resources and its Projects. Technologically it supports the Company's Management Bodies,
and Human Resources on Internal ICT Infrastructure.
It processes the Company's supplies.
It supports the Company in managing all funded projects.
It carries out the accounting management and financial monitoring of the Company and the financed Projects.

\subsection{Directorate of Electronic Infrastructure}
The Directorate of Electronic Infrastructure has the responsibility to operate, upgrades, install, modify
and supervise the performance of the productive infrastructure and services,
maintenance and costing of its operation, the management of the GRNET support center and infrastructure support teams,
contacts with contractors, suppliers and subcontractors.

\subsection{Information Services and Application Development Division}
The Information Services and Application Development Division is responsible for the full life cycle of the projects
and applications deployed internally in the company, from the planning phase to the maintenance process.
It also owns the information services that the company has to the users of the wider educational and research community.
It helps define the company's strategy in the field of ICT and its implementation.
It also participates in the submission of selected proposals and in the execution of national and European projects.



\subsection{My Role}
In GRNET my role is that of a backend developer. As a backend developer I primarily design and
develop APIs. Specifically, I worked a project called Dilosi. The Project Dilosi is a project carried out by GRNET on behalf of the Greek
Government and its main purpose is digitalize the process of issuing a \textbf{solemn
declaration} as well as an \textbf{authorization}. This project aims to change the way
Greek citizens go about issuing a formal declaration which is going to KEP.
There are over 10 developers working on this project. Out of these developers some
of them are backend developers, some are frontend and the rest are responsible for the
CI/CD(Continuous Integration / Continuous Development), scaling and deployment of the app. Dilosi started on early October and it is scheduled
to go live by the end of March. Regarding the technologies used on this project, the frontend is
developed using REACT Framework ~\citep{react} while the backend is based on Django Framework~\citep{django}. As for the CI/CD as well as our Version Control System we use
our own GitLab server. Currently as a database we use PostgreSQL but it is very possible that
we will migrate to Oracle as this is what the government uses currently.
For building the application as a whole (frontend, backend, database) we
use Docker. Regarding the way we operate, we use Scrum which is an agile process
framework for managing complex knowledge work. Scrum consist of daily meetings in our subgroups,
as well as weekly meetings. Every couple of weeks we schedule a retrospective event where we discuss
how the past weeks were, what was wrong, what was right, what to fix and propose ways to strengthen
the team.

\subsubsection{Required Skills}
In order to be a backed developer I had to be able to perform well under pressure. I had to be able
to produce high quality, maintainable and reusable code using Python Language. I had to be able to develop
RESTful web services, have good knowledge of UNIX based systems and be familiar enough with Git.
Moreover, I had to be able to communicate ideas, meet deadlines and overall be an active team
member. Furthermore, I needed to understand and cooperate with developers from other subgroups
such as frontend developers. Being able to use the terminal efficiently is a big must as well as be familiar with concepts such as ssh and public/private keys. Another important requirement is to be
able to use third party libraries and be able to read through their documentation to see what fits
the needs of the project.

\section{Tasks}
In Dilosi I was assigned many tasks all of which I completed successfully. Specifically, 
\begin{itemize}
\item I assisted in designing the database models as well as most of the API.
\item I designed programatically the template of the new digital solemn declaration and authorization based on the original paper ones.
\item I developed the logic of creating/submitting a Declaration.
\item I developed the SMS Verification functionality.
\item I designed and implemented the My / Entity Inbox which is the inbox of either an agent that has a pending declaration or a citizen that has been authorized by someone else.
\item I designed partially and implemented the Actions feature which basically includes all the possible actions an Entity or a User can perform on a declaration (accept/reject/revoke etc.)
\item I implemented the backend infrastructure for administrators to add / edit new Declaration Templates e.g Zero Income Declaration Template and Template Types e.g., Editing the "Solemn Declaration" Template Type footnotes.
\item I have written most of the backend tests which are mostly integration tests.
\item  I have also implemented the sharing functionality declarations and authorizations have in order for agencies and citizens to receive them in their inboxes.
\item I have developed all the action logging infrastructure of the backend.
\end{itemize}


\section{Results}
Overall my results were pretty good which is the main reason I was hired as a full time 
backend developer for various national projects such as the platform for digital driver license
issuing or the Pre School enrollment platform. Specifically, regarding the above tasks all the 
aforementioned featured I was asked to implement have now been merged in our development branch
and are about to go into production one Dilosi goes live. Three of the major tasks I has to complete
were the new digital templates of solemn declaration and authorization, all the backend API tests and
and the implementation of My/Entity Inbox.

\subsubsection{Design of digital solemn declaration and authorization template}
This was one very important and challenging tasks as I had to first print the original
paper application an take a ruler, measure each box, its margins etc. and try and recreate
the same design digitally. For this project I worked along with Prof. Panagiotis Louridas at
a separate repository from the rest of dilosi project. The newly created template compared to the
original ones should also include a qr code containing the reference code of the application as well
as being digitally sealed with the digital government seal. Another challenge I face was to implement
a way to shrink the text to fit the boxes as I couldn't the text length anyone would insert so I had to this dynamically. I have spent over 50 hours over the course of 20 days on this task and it is being maintained until now with various improvements and bugfixes. As mentioned before this is an extremely important task not only for the company and the projects but for all of the Greek citizens, 
as without it after completing a digital solemn declaration fields nothing would happen. Fortunately, there were no delays compared to the initial planning. Everything went smooth and this piece of code is currently merged into the development branch. The way I worked was by pushing commits everyday through Git tou GitHub initially and the merge it to GitLab.

\subsubsection{API Tests}
Another extremely import task I managed to complete was the develop the full testing infrastructure
of the API. Specifically, this was most probably the most important and longest in duration task I was
assigned. What I had to do was to write integration tests to test the flow of the application as a whole. Unfortunately. this is not a one time task. As more code and features get added to the application more tests have to be written to ensure that the application, features and their functions
operate as intended. Imagine what would happen if there were no tests and our untested application would go into production and being used by 11 million Greek citizens? Therefore I consider this task
extremely important. Regarding the task itself I have written over 1000 lines of code and about 25 tests with multiple functions each, which test from the creation of a declaration to accepting or rejecting an inbox. The special thing about these test is that they are not unit tests but integration tests which means that they connect to each other testing a use case and a flow. Again on this task I commited every day through Git to GitLab. I have spent on this task over 100 hours in a course of my 4 month internship and I expect to spent many more hours as code gets integrated in our codebase. Some challenges I faced were that I had to go through the Django testing documentation in order to write the script that would execute the tests as well as trying to get the coverage as high as possible. Some issues that emerged were that when the CI/CD team tried to integrate our test suite to GitLab so it runs every time someone pushes a commit, they couldn't do it successfully do the test having third party dependencies mostly with memcache so  I had to refactor many things in our tests in order to make it work.


\subsubsection{My / Entity Inbox}
This task is a fairly recent task I was assigned. It basically involved creating all the backend infrastructure for different government entities e.g., DOI, EFKA etc. as well as citizens to be able
to be notified about incoming declaration and authorization as well as being able to interact with them. Something like an email inbox but for declarations and authorizations. Again this is a very important task as without it there wouldn't be any point in having digital declarations if there was not a way to send them accordingly. The way this feature operates is that when someone picks a recipient while completing the fields of the declaration there are 3 scenarios. First scenario is that he picks a recipient from the list of verified recipients. These are agencies that have been verified by the Greek government. In that case the declaration automatically goes to the Inbox of that recipient as has a pending state. Second scenario is picking an entity only by its name. In that case since we cannot easily assign this name to a verified entity due to conflicts in the name (lowercase letters, uppercase letters, spelling mistakes etc.) we create the declaration but won't put it to anyone's inbox. Third scenario is that someone picks an entity by both its name and email or just email. In that case although the entity might not exist and to being verified having the unique email is a way to create that entity's inbox and put that declaration there. At the same time we notify that email that there is a pending declaration for them and when the entity logs in to review the declaration then it will get verified. This case usually refers to schools and a popular declaration would be the No Religion declaration many parent do for their kids. Regarding this task
i have spent about 80 hours over the course of about 5 days. Again, the way I worked was to push commits everyday to our application repository in GitLab. Some challenges I face was the how to design the feature in order to fulfill most cases and scenarios as well as how to develop it efficiently. Of course this feature came with its own set of tests which were added to our tests suite. Regarding the initial scheduling there were 3 days of delay due to me being sick at that time.

\section{Time / Project Scheduling}
In the following table I mention the task I was assigned as discussed earlier along with the
duration of each one in days. I should mention that if a task states a number of days it doesn't mean
that I worked on this task for X consecutive days, but instead it could be a sum of multiple hours.
A day is considered to be 8 hours. Some durations have an asterisk which means that although this task is initially completed it might get improvements and bugfixes.

% Please add the following required packages to your document preamble:
% \usepackage{longtable}
% Note: It may be necessary to compile the document several times to get a multi-page table to line up properly
\begin{longtable}[c]{|l|l|}
	\hline
	\multicolumn{1}{|c|}{\textbf{Activity / Task}} & \multicolumn{1}{c|}{\textbf{Duration in Days}} \\ \hline
	\endfirsthead
	%
	\endhead
	%
	Designing of Models / Database Schema & 4 \\ \hline
	\begin{tabular}[c]{@{}l@{}}Designing and Developing Digital Solemn Declaration\\ and Authorization pdf Templates\end{tabular} & 20* \\ \hline
	Developing of API Tests & 100* \\ \hline
	Implementation of the SMS Verification System & 3 \\ \hline
	Implementation of the Creation/Submission of a Solemn Declaration & 17 \\ \hline
	Implementation of Sharing Functionality & 2 \\ \hline
	Refactoring Codebase & 40* \\ \hline
	Desging and Implementation of  My / Entity Inbox basic flows & 10 \\ \hline
	Design and Implementation of  Templates / Template Types / Fields & 63* \\ \hline
	Database Stress Test Script & 3 \\ \hline
	Implementation of Entity / User Actions & 32 \\ \hline
	\caption{Internship Activities / Tasks along with their duration}
	\label{InternshipTasks}
\end{longtable}

\begin{figure}

\begin{center}

\begin{ganttchart}[y unit title=0.6cm,
	y unit chart=0.8cm,
	vgrid,
	hgrid,
	title label anchor/.style={below=-1.6ex},
	title left shift=.05,
	title right shift=-.05,
	title height=1,bar/.style={fill=gray!50},
	incomplete/.style={fill=white},
	progress label text={},
	bar height=0.7,group right shift=0,
	group top shift=.6,
	group height=.3]{1}{24}
\gantttitle{Internship}{24} \\
\gantttitle{1/10}{3} 
\gantttitle{26/10}{3}
\gantttitle{10/11}{3}
\gantttitle{25/11}{3} 
\gantttitle{19/12}{3}
\gantttitle{15/1}{3}
 \gantttitle{1/2}{3}
\gantttitle{20/2}{3} \\
%tasks
\ganttbar{DB Schema}{1}{2} \\
\ganttbar{Pdf templates}{6}{12} \\
\ganttbar{API Tests}{3}{24} \\
\ganttbar{SMS Verification}{4}{5} \\
\ganttbar[progress=33]{Create/Submit Declaration}{6.5}{10} \\
\ganttbar{Sharing Functionality}{22}{24} \\
\ganttbar{Refactor Codebase}{4}{24} \\
\ganttbar{Design My/Entity Inbox}{19}{24} \\
\ganttbar{Template/Types}{12}{19} \\
\ganttbar{DB Stress Script}{23}{24} \\
\ganttbar{Entity/User Actions}{14}{24}

\end{ganttchart}
\end{center}
\caption{Internship Tasks Gantt Chart}
\end{figure}

\pagebreak

\section{Skills}
During my internship I used many skills which I learned from AUEB while being an undergraduate student. In the table that follows I present those skills along with the course that taught me, the methods and some example in which is used them.

% Please add the following required packages to your document preamble:
% \usepackage[normalem]{ulem}
% \useunder{\uline}{\ul}{}
% \usepackage{longtable}
% Note: It may be necessary to compile the document several times to get a multi-page table to line up properly
\begin{longtable}[c]{|l|l|l|}
	\hline
	\textbf{Skills} & \textbf{Method} & \textbf{Example} \\ \hline
	\endfirsthead
	%
	\endhead
	%
	\begin{tabular}[c]{@{}l@{}}Database Management\\ Systems\end{tabular} & Microsoft SQL Server & \begin{tabular}[c]{@{}l@{}}I used this skill to query the SQLite\\ and PostreSQL Database of dilosi \\ application.\end{tabular} \\ \hline
	\begin{tabular}[c]{@{}l@{}}Software Engineering\\ in Practise\end{tabular} & Software Design Patterns, & \begin{tabular}[c]{@{}l@{}}Developed high quality, maintainable,\\ modular and testable code\end{tabular} \\ \hline
	Cloud Applications & Django Framework & \begin{tabular}[c]{@{}l@{}}Since Dilosi is based on Django\\ Framework,  this was the most \\ important skill. All the code I\\ have written the past months is \\ within Django context\end{tabular} \\ \hline
	Project Management & Work Scheduling & \begin{tabular}[c]{@{}l@{}}It helped me by prioritize\\ work in such a way so \\ there are no delays\end{tabular} \\ \hline
	Programming 2 & JUnit, Git & \begin{tabular}[c]{@{}l@{}}I used git every day to \\ publish my work as well\\ as Integration tests to test\\ the API.\end{tabular} \\ \hline
	\begin{tabular}[c]{@{}l@{}}Personal Skills\\ Development\end{tabular} & \begin{tabular}[c]{@{}l@{}}Interpersonal \\ Communication\end{tabular} & \begin{tabular}[c]{@{}l@{}}It helped my communicate\\ better with my coworkers\\ and understand each other\\ while being respectful and\\ tolerant.\end{tabular} \\ \hline
	\begin{tabular}[c]{@{}l@{}}Optimization Methods in\\ Management Science\end{tabular} & \begin{tabular}[c]{@{}l@{}}Analysis and design of\\ optimization methods\end{tabular} & \begin{tabular}[c]{@{}l@{}}It helped to better schedule\\ my assignments.\end{tabular} \\ \hline
	\begin{tabular}[c]{@{}l@{}}Algorithms and Data \\ Structures\end{tabular} & Python, Algorithms & \begin{tabular}[c]{@{}l@{}}I used Python as my main\\ development Language\\ every day.\end{tabular} \\ \hline
		\caption{University Skills that proved useful}
		\label{UniversitySkilss}
\end{longtable}



\section{Comments}
The past 4 months have been great. From day one I felt very welcomed in GRNET. The on-boarding package included  a desk to work and MacBook Pro along with an external monitor, a parking slot, lunch from
a catering and a very good salary. All developers I worked with were very welcoming and eager to help, respectful and tolerant to mistakes. The department I worked within is very well organized, with weekly team meetings, 1-1 meetings with the supervisor and monthly written report. Every couple months we host a retrospective event where we share our thought for the past months and propose ways to become better at a personal and team level. At first I was anxious and stressed that I wouldn't be able to perform good enough bearing in mind that my coworkers are very experienced and have worked in great companies in the past but as very soon I came to understood that this wasn't the fact because what I was assigned to do is exactly what I have been taught all these years in university so in a sense I was preparing for the past 4 years for this job. The past four months for me have been mostly learning new things ranging from technologies, to coding languages, to developing new interpersonal skills to understanding how a working environment operates. I have master my GIT skills, I have started developing in VIM Editor, I have familiarized with Docker, Kubernetes and CI/CD. If I had to change something in my Department I would change the fact that there isn't a single source of truth. We use at least 3 platform to post our issues and our TODOs
and there is not much vertical communication in terms of requirements and scheduling. I would rather propose having a single platform where everything is posted there from issues to news regarding the Greek government requirements and so on. Something else i would like to point out as I believe it is very important especially for developers is their work to be widely used. At GRNET I had a very rare opportunity of developing application for the Greek government, in other words an application that will be used by 11 million Greek citizens. Overall, I am extremely happy and grateful for this opportunity and once again I would like to thanks my university for preparing me accordingly all these years, Prof. Panagiotis Louridas for supporting me every step of the way, my coworkers for welcoming me and helping me everyday to become a better software engineer and AUEB's Internship Office for providing me with the opportunity to work at such a great company.


